% Данный файл распространяетсяя под лицензией CC BY 4.0. Текст лицензии размещён на https://creativecommons.org/licenses/by/4.0/
% (c) Кафедра системного программирования, 2025

\documentclass[a4paper]{article}

\usepackage[a4paper, top=8mm, bottom=8mm, left=8mm, right=8mm]{geometry}

\usepackage{polyglossia}
\setdefaultlanguage[babelshorthands=true]{russian}
\setotherlanguage{english}

\usepackage{fontspec}
\setmainfont{FreeSerif}
\newfontfamily{\russianfonttt}[Scale=0.7]{DejaVuSansMono}

\usepackage[tiny, compact]{titlesec}

\usepackage{titling}
\setlength{\droptitle}{-1cm}
\pretitle{\begin{center}\begin{bfseries}\Large}
\posttitle{\par\end{bfseries}\end{center}}
\preauthor{\begin{center}\normalsize}
\postauthor{\par\end{center}\vspace{-1.8cm}}

\usepackage{hyperref}
\usepackage{bookmark}
\usepackage{csquotes}
% Сюда можно дописывать нужные вам пакеты.

\title{Название работы}

\author{ФИО}

% Дата много места занимает, её лучше не указывать.
\date{}

\begin{document}

\maketitle

\begin{flushright}
    Группа: \emph{группа в формате \enquote{<год поступления>.<Б/М><номер группы>-мм}}

    Кафедра: \emph{кафедра, на которой планируется защита/сдача работы}

    % Степень/звание и должность научника мы лучше вас знаем, их можно не указывать.
    Научный руководитель: \emph{ФИО научного руководителя}

    % А вот про консультанта укажите официальное место работы, с "ООО" или чем-то таким, и должностью желательно по трудовой, не просто "techlead". Выясните у консультанта.
    Консультант: \emph{место работы, должность, степень, звание (если есть) консультанта}
    
    Номер семестра практики: \emph{семестр, за который практика}
\end{flushright} 


\section{Постановка задачи}

% Буквально парой предложений ввести в контекст, обязательно сформулировать цель работы. Если работа делается где-то глубоко внутри проекта, на введение можно потратить больше места, чтобы подвести читателя от сути проекта в целом к тому, что конкретно вам надо было сделать. Только без воды, никакого "в современном мире"!
Работа выполняется в компании \dots в рамках проекта по \dots. \textbf{Целью работы является} \dots

% Это пояснение актуальности. Тут обязательно надо обрисовать полезность вашей работы для конечного пользователя (и заодно как-то описать, кто такой ваш конечный пользователь --- может, это соседняя команда). Даже если вы делаете работу "глубоко внутри", выясните у консультанта, чем это людям поможет. И опишите, кто выступал в роли "заказчика" вашей работы, т.е. кто и зачем предложил задачу. Можно без конкретных имён, но названия организаций вполне желательны.
Результаты работы необходимы для \dots, для конечных пользователей продукта это полезно тем-то, в выполнении работы заинтересованы те-то.

\section{Описание предлагаемого решения}

% Раздел пишется в свободной форме, с минимумом технических подробностей (можно приводить ссылки на более подробные документы).

Тут надо кратко описать идею решения, спозиционировать его относительно предыдущих наработок\footnote{Если на них надо сослаться, то в подстраничной сноске, URL: \url{http://example.com} (дата обращения: не забываем указывать).} и существующих аналогов (тоже предельно кратко), по возможности обосновать принятые решения.

% Вот это важно, опишите кратко, что использовали
Реализация решения осуществлялась с использованием \dots (тут описать используемые технологии).

\section{Апробация / Эксперименты}

% Название секции можно поменять на более подходящее вашей работе

Название этой секции --- либо \enquote{Апробация}, либо \enquote{эксперименты} (если содержательно есть и то и то, можно две отдельные секции сделать). Тут опишите, как проводилась проверка того, что получилось что-то адекватное. Если есть отзывы пользователей, отлично (особенно если они в структурированном виде, типа анкеты SUS). Если есть отзывы команды, консультанта и т.п., хорошо. Если есть только тесты. то хотя бы их тут опишите.

Если работа подразумевала замеры чего-то (например, производительности), приведите тут итоговые результаты и выводы, и дайте ссылку на таблицы с данными. Не забывайте про матстат (приводите матожидание и среднеквадратичное отклонение, не просто результат замера).

Приведите выводы из эксперимента.

\section{Заключение}

В ходе выполнения работы получены следующие результаты:

% ниже списком перечисляете то, что выносится как результат практики.

\begin{itemize}
    \item \dots
    \item \dots
    \item \dots
\end{itemize}

Материалы, иллюстрирующие выполнение работы, размещены по адресу: \url{http://www.example.com} (если есть и применимо --- скриншоты, видео на 1-2 минуты и т.п. с демонстрацией; может быть очень полезно даже для консольных утилит или библиотек).

Репозиторий/ссылка на пуллреквест: \url{http://www.example.com} (это обязательно; ссылок может быть несколько).

Техническая документация: \url{http://www.example.com} (если применимо --- для пуллреквестов можно техническую сторону прямо в комментарии к пуллреквесту указать, но можно и в отдельную вики-страницу вынести).

Решение внедрено в производственные процессы компании/прошло процесс ревью/находится на ревью/находится в разработке и т.п. В общем, кратко сформулируйте, что в итоге получилось и в каком статусе результат, чтобы не было ощущения, что оно сделано, положено на GitHub и забыто.

\end{document}